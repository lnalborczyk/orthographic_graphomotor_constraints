% Options for packages loaded elsewhere
\PassOptionsToPackage{unicode}{hyperref}
\PassOptionsToPackage{hyphens}{url}
%
\documentclass[
  11pt,
  english,
  ,doc,floatsintext]{apa6}
\title{Supplementary materials to `Interaction between orthographic and graphomotor constraints in learning to write'}
\author{Jérémy Danna\textsuperscript{1}, Marieke Longcamp\textsuperscript{1}, Ladislas Nalborczyk\textsuperscript{1, 2}, Jean-Luc Velay\textsuperscript{1}, Claire Commengé\textsuperscript{2}, \& Marianne Jover\textsuperscript{3}}
\date{}

\usepackage{amsmath,amssymb}
\usepackage{lmodern}
\usepackage{setspace}
\usepackage{iftex}
\ifPDFTeX
  \usepackage[T1]{fontenc}
  \usepackage[utf8]{inputenc}
  \usepackage{textcomp} % provide euro and other symbols
\else % if luatex or xetex
  \usepackage{unicode-math}
  \defaultfontfeatures{Scale=MatchLowercase}
  \defaultfontfeatures[\rmfamily]{Ligatures=TeX,Scale=1}
\fi
% Use upquote if available, for straight quotes in verbatim environments
\IfFileExists{upquote.sty}{\usepackage{upquote}}{}
\IfFileExists{microtype.sty}{% use microtype if available
  \usepackage[]{microtype}
  \UseMicrotypeSet[protrusion]{basicmath} % disable protrusion for tt fonts
}{}
\makeatletter
\@ifundefined{KOMAClassName}{% if non-KOMA class
  \IfFileExists{parskip.sty}{%
    \usepackage{parskip}
  }{% else
    \setlength{\parindent}{0pt}
    \setlength{\parskip}{6pt plus 2pt minus 1pt}}
}{% if KOMA class
  \KOMAoptions{parskip=half}}
\makeatother
\usepackage{xcolor}
\IfFileExists{xurl.sty}{\usepackage{xurl}}{} % add URL line breaks if available
\IfFileExists{bookmark.sty}{\usepackage{bookmark}}{\usepackage{hyperref}}
\hypersetup{
  pdftitle={Supplementary materials to `Interaction between orthographic and graphomotor constraints in learning to write'},
  pdfauthor={Jérémy Danna1, Marieke Longcamp1, Ladislas Nalborczyk1, 2, Jean-Luc Velay1, Claire Commengé2, \& Marianne Jover3},
  pdflang={en-EN},
  hidelinks,
  pdfcreator={LaTeX via pandoc}}
\urlstyle{same} % disable monospaced font for URLs
\usepackage{color}
\usepackage{fancyvrb}
\newcommand{\VerbBar}{|}
\newcommand{\VERB}{\Verb[commandchars=\\\{\}]}
\DefineVerbatimEnvironment{Highlighting}{Verbatim}{commandchars=\\\{\}}
% Add ',fontsize=\small' for more characters per line
\newenvironment{Shaded}{}{}
\newcommand{\AlertTok}[1]{\textcolor[rgb]{1.00,0.00,0.00}{\textbf{#1}}}
\newcommand{\AnnotationTok}[1]{\textcolor[rgb]{0.38,0.63,0.69}{\textbf{\textit{#1}}}}
\newcommand{\AttributeTok}[1]{\textcolor[rgb]{0.49,0.56,0.16}{#1}}
\newcommand{\BaseNTok}[1]{\textcolor[rgb]{0.25,0.63,0.44}{#1}}
\newcommand{\BuiltInTok}[1]{#1}
\newcommand{\CharTok}[1]{\textcolor[rgb]{0.25,0.44,0.63}{#1}}
\newcommand{\CommentTok}[1]{\textcolor[rgb]{0.38,0.63,0.69}{\textit{#1}}}
\newcommand{\CommentVarTok}[1]{\textcolor[rgb]{0.38,0.63,0.69}{\textbf{\textit{#1}}}}
\newcommand{\ConstantTok}[1]{\textcolor[rgb]{0.53,0.00,0.00}{#1}}
\newcommand{\ControlFlowTok}[1]{\textcolor[rgb]{0.00,0.44,0.13}{\textbf{#1}}}
\newcommand{\DataTypeTok}[1]{\textcolor[rgb]{0.56,0.13,0.00}{#1}}
\newcommand{\DecValTok}[1]{\textcolor[rgb]{0.25,0.63,0.44}{#1}}
\newcommand{\DocumentationTok}[1]{\textcolor[rgb]{0.73,0.13,0.13}{\textit{#1}}}
\newcommand{\ErrorTok}[1]{\textcolor[rgb]{1.00,0.00,0.00}{\textbf{#1}}}
\newcommand{\ExtensionTok}[1]{#1}
\newcommand{\FloatTok}[1]{\textcolor[rgb]{0.25,0.63,0.44}{#1}}
\newcommand{\FunctionTok}[1]{\textcolor[rgb]{0.02,0.16,0.49}{#1}}
\newcommand{\ImportTok}[1]{#1}
\newcommand{\InformationTok}[1]{\textcolor[rgb]{0.38,0.63,0.69}{\textbf{\textit{#1}}}}
\newcommand{\KeywordTok}[1]{\textcolor[rgb]{0.00,0.44,0.13}{\textbf{#1}}}
\newcommand{\NormalTok}[1]{#1}
\newcommand{\OperatorTok}[1]{\textcolor[rgb]{0.40,0.40,0.40}{#1}}
\newcommand{\OtherTok}[1]{\textcolor[rgb]{0.00,0.44,0.13}{#1}}
\newcommand{\PreprocessorTok}[1]{\textcolor[rgb]{0.74,0.48,0.00}{#1}}
\newcommand{\RegionMarkerTok}[1]{#1}
\newcommand{\SpecialCharTok}[1]{\textcolor[rgb]{0.25,0.44,0.63}{#1}}
\newcommand{\SpecialStringTok}[1]{\textcolor[rgb]{0.73,0.40,0.53}{#1}}
\newcommand{\StringTok}[1]{\textcolor[rgb]{0.25,0.44,0.63}{#1}}
\newcommand{\VariableTok}[1]{\textcolor[rgb]{0.10,0.09,0.49}{#1}}
\newcommand{\VerbatimStringTok}[1]{\textcolor[rgb]{0.25,0.44,0.63}{#1}}
\newcommand{\WarningTok}[1]{\textcolor[rgb]{0.38,0.63,0.69}{\textbf{\textit{#1}}}}
\usepackage{graphicx}
\makeatletter
\def\maxwidth{\ifdim\Gin@nat@width>\linewidth\linewidth\else\Gin@nat@width\fi}
\def\maxheight{\ifdim\Gin@nat@height>\textheight\textheight\else\Gin@nat@height\fi}
\makeatother
% Scale images if necessary, so that they will not overflow the page
% margins by default, and it is still possible to overwrite the defaults
% using explicit options in \includegraphics[width, height, ...]{}
\setkeys{Gin}{width=\maxwidth,height=\maxheight,keepaspectratio}
% Set default figure placement to htbp
\makeatletter
\def\fps@figure{htbp}
\makeatother
\setlength{\emergencystretch}{3em} % prevent overfull lines
\providecommand{\tightlist}{%
  \setlength{\itemsep}{0pt}\setlength{\parskip}{0pt}}
\setcounter{secnumdepth}{5}
% Make \paragraph and \subparagraph free-standing
\ifx\paragraph\undefined\else
  \let\oldparagraph\paragraph
  \renewcommand{\paragraph}[1]{\oldparagraph{#1}\mbox{}}
\fi
\ifx\subparagraph\undefined\else
  \let\oldsubparagraph\subparagraph
  \renewcommand{\subparagraph}[1]{\oldsubparagraph{#1}\mbox{}}
\fi
\newlength{\cslhangindent}
\setlength{\cslhangindent}{1.5em}
\newlength{\csllabelwidth}
\setlength{\csllabelwidth}{3em}
\newlength{\cslentryspacingunit} % times entry-spacing
\setlength{\cslentryspacingunit}{\parskip}
\newenvironment{CSLReferences}[2] % #1 hanging-ident, #2 entry spacing
 {% don't indent paragraphs
  \setlength{\parindent}{0pt}
  % turn on hanging indent if param 1 is 1
  \ifodd #1
  \let\oldpar\par
  \def\par{\hangindent=\cslhangindent\oldpar}
  \fi
  % set entry spacing
  \setlength{\parskip}{#2\cslentryspacingunit}
 }%
 {}
\usepackage{calc}
\newcommand{\CSLBlock}[1]{#1\hfill\break}
\newcommand{\CSLLeftMargin}[1]{\parbox[t]{\csllabelwidth}{#1}}
\newcommand{\CSLRightInline}[1]{\parbox[t]{\linewidth - \csllabelwidth}{#1}\break}
\newcommand{\CSLIndent}[1]{\hspace{\cslhangindent}#1}
% Manuscript styling
\usepackage{upgreek}
\captionsetup{font=singlespacing,justification=justified}

% Table formatting
\usepackage{longtable}
\usepackage{lscape}
% \usepackage[counterclockwise]{rotating}   % Landscape page setup for large tables
\usepackage{multirow}		% Table styling
\usepackage{tabularx}		% Control Column width
\usepackage[flushleft]{threeparttable}	% Allows for three part tables with a specified notes section
\usepackage{threeparttablex}            % Lets threeparttable work with longtable

% Create new environments so endfloat can handle them
% \newenvironment{ltable}
%   {\begin{landscape}\centering\begin{threeparttable}}
%   {\end{threeparttable}\end{landscape}}
\newenvironment{lltable}{\begin{landscape}\centering\begin{ThreePartTable}}{\end{ThreePartTable}\end{landscape}}

% Enables adjusting longtable caption width to table width
% Solution found at http://golatex.de/longtable-mit-caption-so-breit-wie-die-tabelle-t15767.html
\makeatletter
\newcommand\LastLTentrywidth{1em}
\newlength\longtablewidth
\setlength{\longtablewidth}{1in}
\newcommand{\getlongtablewidth}{\begingroup \ifcsname LT@\roman{LT@tables}\endcsname \global\longtablewidth=0pt \renewcommand{\LT@entry}[2]{\global\advance\longtablewidth by ##2\relax\gdef\LastLTentrywidth{##2}}\@nameuse{LT@\roman{LT@tables}} \fi \endgroup}

% \setlength{\parindent}{0.5in}
% \setlength{\parskip}{0pt plus 0pt minus 0pt}

% \usepackage{etoolbox}
\makeatletter
\patchcmd{\HyOrg@maketitle}
  {\section{\normalfont\normalsize\abstractname}}
  {\section*{\normalfont\normalsize\abstractname}}
  {}{\typeout{Failed to patch abstract.}}
\patchcmd{\HyOrg@maketitle}
  {\section{\protect\normalfont{\@title}}}
  {\section*{\protect\normalfont{\@title}}}
  {}{\typeout{Failed to patch title.}}
\makeatother
\shorttitle{Supplementary Materials}
\usepackage{csquotes}
\usepackage{float}
\ifXeTeX
  % Load polyglossia as late as possible: uses bidi with RTL langages (e.g. Hebrew, Arabic)
  \usepackage{polyglossia}
  \setmainlanguage[]{english}
\else
  \usepackage[main=english]{babel}
% get rid of language-specific shorthands (see #6817):
\let\LanguageShortHands\languageshorthands
\def\languageshorthands#1{}
\fi
\ifLuaTeX
  \usepackage{selnolig}  % disable illegal ligatures
\fi


\affiliation{\vspace{0.5cm}\textsuperscript{1} Aix Marseille Univ, CNRS, LNC, Marseille, France\\\textsuperscript{2} Aix Marseille Univ, CNRS, LPC, Marseille, France\\\textsuperscript{3} Aix Marseille Univ, PSYCLE, Aix-en-Provence, France}

\begin{document}
\maketitle

{
\setcounter{tocdepth}{3}
\tableofcontents
}
\setstretch{1.25}
\newpage

\hypertarget{visual-data-exploration}{%
\section{Visual data exploration}\label{visual-data-exploration}}

\hypertarget{univariate-raw-data-exploration}{%
\subsection{Univariate raw data exploration}\label{univariate-raw-data-exploration}}

\begin{figure}[!htb]

{\centering \includegraphics[width=1\linewidth]{supplementary_materials_files/figure-latex/data-1} 

}

\caption{Effect of grade, word frequency, and grapheme complexity on the total duration, the number of stops, the mean velocity, and the letter size. The error bars represent the 95\% confidence intervals of the mean (assuming a Gaussian distribution).}\label{fig:data}
\end{figure}

Figure \ref{fig:data} shows the effect of grade, word frequency, and grapheme complexity on the letter duration, the number of stops, the mean velocity, and the letter size. This figure suggests that the average duration (in seconds) seems to decrease monotonically with grade. The number of stops also seems to decrease with grade, with most trials for children from grade 2 being associated with no stop.

\hypertarget{bivariate-correlations-by-grade}{%
\subsection{Bivariate correlations by grade}\label{bivariate-correlations-by-grade}}

Figure \ref{fig:correlation} shows the overall and by-grade Spearman correlation between each pair of variables. This figure reveals medium to strong positive and negative correlations between each pair of variable. These relations are sometimes non-linear (e.g., between duration and mean velocity), hence the use of Spearman (rather than Pearson) correlation coefficients.

\begin{figure}[!htb]

{\centering \includegraphics[width=1\linewidth]{supplementary_materials_files/figure-latex/correlation-1} 

}

\caption{Overall and by-grade Spearman correlation between each pair of measured variables.}\label{fig:correlation}
\end{figure}

\newpage

\hypertarget{bayesian-multilevel-modelling}{%
\section{Bayesian multilevel modelling}\label{bayesian-multilevel-modelling}}

\hypertarget{modelling-positive-only-values}{%
\subsection{Modelling positive-only values}\label{modelling-positive-only-values}}

A dominant feature of durations or response times (RTs) is that their distribution is generally positively skewed, with the spread and/or the skewness increasing with task difficulty (for review, see for instance \protect\hyperlink{ref-forstmann_sequential_2016}{Forstmann, Ratcliff, \& Wagenmakers, 2016}). Therefore, several models have been proposed to account for the peculiarities of the data coming from such tasks as well as to relate it to the underlying cognitive processes. We discuss below why using Gaussian models for this kind of data is generally not a sensible idea and describe our approach in more details. We follow a general ``Bayesian workflow'' by building our model in an iterative manner and by motivating and validating each modelling choice (for more details, see for instance \protect\hyperlink{ref-gelman_bayesian_2020}{Gelman et al., 2020}).

We first fitted a Bayesian multilevel (also known as ``mixed-effects'') Gaussian multivariate (i.e., with multiple outcomes) model. One way of evaluating this model is to evaluate its predictions. If this model is a good description of the process that generated the observed data, then it should be able to generate data that looks like the observed data. The process of generating data from the estimated posterior distribution is called \emph{posterior predictive checking} and can be used in many different ways using the \texttt{pp\_check()} method (\protect\hyperlink{ref-gabry_visualization_2019}{Gabry, Simpson, Vehtari, Betancourt, \& Gelman, 2019}). In Figure \ref{fig:positive}, we depict the distribution of the raw data along with the distribution of 100 simulated datasets.

\begin{figure}[!htb]

{\centering \includegraphics[width=1\linewidth]{supplementary_materials_files/figure-latex/positive-1} 

}

\caption{Posterior predictive checks for the multilevel Gaussian model. The dark blue density depicts the raw data whereas light blue densities represent data simulated from the posterior predictive distribution.}\label{fig:positive}
\end{figure}

This figure reveals that the Gaussian model fails to account for the peculiarities of the data at hand. For instance, it systematically fails to predict the right-skew of all four variables, and more dramatically, sometimes predicts negatives values for these variables, although they are strictly positive. Moreover, using a Gaussian distribution to model the number of stops also leads to nonsensical predictions as the number of stops is necessarily a positive integer (whereas the Gaussian distribution can produce any real number), as illustrated in the upper right panel of Figure \ref{fig:positive}.

\hypertarget{shifted-lognormal-regression-model}{%
\subsection{Shifted-lognormal regression model}\label{shifted-lognormal-regression-model}}

A useful description of RTs or durations should be able to account for the effects of the difficulty of the task, as well as changes in shift and spread of the distribution. The Log-normal, Ex-Gaussian, or Weibull distributions often provide a good fit to these data, but their parameters are difficult to interpret in terms of difficulty, shift, or spread (i.e., these distributions do not have straightforward interpretable parameters). In contrast, the shifted log-normal distribution has parameters that can easily be interpreted in terms of difficulty, shift, and spread.

The log-normal distribution is called ``log-normal'' because the parameters are the mean and standard deviation of the log-transformed response times, which is assumed to be a normal (Gaussian) distribution. The shifted log-normal distribution is then described by three parameters:

\begin{itemize}
\item
  \(\mu\) (mu, difficulty): the mean of the log-normal distribution. The mean of \(\mu\)
  represents the median response time. The median response time is given by \(\text{shift} + \exp (\mu)\).
\item
  \(\sigma\) (sigma, scale): the standard deviation of the log-normal distribution. Increases the mean but not the median of \(\mu\).
\item
  \(\text{shift}\) (ndt) indicates the time of the earliest possible response. When \(\text{shift} = 0\), the shifted log-normal distribution correspond to the conventional log-normal distribution with two parameters.
\end{itemize}

Regression models using this family usually aim to predict \(\mu\), the mean of the log-normal distribution. We employ this strategy as well, while noticing that both \(\sigma\) and \(\text{shift}\) could be allowed to vary across conditions as well.

\hypertarget{poisson-regression-model}{%
\subsection{Poisson regression model}\label{poisson-regression-model}}

Whereas the previous models are able to take into account the right-skew of the four variables we are interested in, they are still not able to make proper predictions with regards to the number of stops (because the log-normal distribution is continuous). Yet a better model could be fitted by picking up a discrete probability distribution defined on the positive integer real. The Poisson regression model is appropriate for modelling discrete counts of events (e.g., the number of stops) that happen in a fixed interval of space or time with no upper bound. The Poisson model is simpler than the Gaussian or the lognormal one because it has only one parameter \(\lambda\) that describes its shape. The parameter \(\lambda\) is the expected value of the outcome \(y\) (and also its expected variance). However, we need a link function to relate the predictors with the parameter \(\lambda\) and to ensure that \(\lambda\) is always positive. We use the conventional logarithmic link function, resulting in the following linear model:

\[
\begin{aligned}
y_{i} &\sim \mathrm{Poisson}(\lambda_{i})\\
\log(\lambda_{i}) &= \alpha + \beta_{g} \cdot \text{grade}_{i} + \beta_{f} \cdot \text{frequency}_{i} +\\ &\quad  \beta_{graphe} \cdot \text{grapheme}_{i} + \beta_{grapho} \cdot \text{graphomotor}_{i}\\
\end{aligned}
\]

This kind of model is now able to predict valid number of stops (i.e., positive integers). Note that for simplicity, we omit the varying effects and the priors from the above model (for more details on Poisson regression, see \protect\hyperlink{ref-winter_poisson_2021}{Winter \& Bürkner, 2021}).

\hypertarget{fitting-the-final-model}{%
\subsection{Fitting the final model}\label{fitting-the-final-model}}

To set up the model, we need to invoke the \texttt{brms::brmsformula()} function and construct one formula for each of the four dependant variables. We fitted all models using the \texttt{brms} package (\protect\hyperlink{ref-R-brms_a}{Bürkner, 2017}) (for an introduction to Bayesian multilevel modelling in brms, see \protect\hyperlink{ref-nalborczyk_introduction_2019}{Nalborczyk, Batailler, Lœvenbruck, Vilain, \& Bürkner, 2019}). We used sum contrasts (i.e., recoding conditions as -0.5 vs.~0.5) for binary predictors (i.e., frequency, grapheme complexity, and graphomotor difficulty) and used the default factor coding scheme (i.e., dummy coding) for grade.

\begin{Shaded}
\begin{Highlighting}[]
\CommentTok{\# defining the model formula for the generalised multilevel model}
\NormalTok{formula\_generalised }\OtherTok{\textless{}{-}}
    \FunctionTok{bf}\NormalTok{(}
\NormalTok{        duration }\SpecialCharTok{\textasciitilde{}} \DecValTok{1} \SpecialCharTok{+}\NormalTok{ group }\SpecialCharTok{*}\NormalTok{ frequency }\SpecialCharTok{*}\NormalTok{ grapheme\_complexity }\SpecialCharTok{*}
\NormalTok{            graphomotor\_difficulty }\SpecialCharTok{+}\NormalTok{ (}\DecValTok{1} \SpecialCharTok{|}\NormalTok{ subject),}
        \AttributeTok{family =} \FunctionTok{shifted\_lognormal}\NormalTok{()}
\NormalTok{        ) }\SpecialCharTok{+}
    \FunctionTok{bf}\NormalTok{(}
\NormalTok{        mean\_velocity }\SpecialCharTok{\textasciitilde{}} \DecValTok{1} \SpecialCharTok{+}\NormalTok{ group }\SpecialCharTok{*}\NormalTok{ frequency }\SpecialCharTok{*}\NormalTok{ grapheme\_complexity }\SpecialCharTok{*}
\NormalTok{            graphomotor\_difficulty }\SpecialCharTok{+}\NormalTok{ (}\DecValTok{1} \SpecialCharTok{|}\NormalTok{ subject),}
        \AttributeTok{family =} \FunctionTok{shifted\_lognormal}\NormalTok{()}
\NormalTok{        ) }\SpecialCharTok{+}
    \FunctionTok{bf}\NormalTok{(}
\NormalTok{        number\_of\_stops }\SpecialCharTok{\textasciitilde{}} \DecValTok{1} \SpecialCharTok{+}\NormalTok{ group }\SpecialCharTok{*}\NormalTok{ frequency }\SpecialCharTok{*}\NormalTok{ grapheme\_complexity }\SpecialCharTok{*}
\NormalTok{            graphomotor\_difficulty }\SpecialCharTok{+}\NormalTok{ (}\DecValTok{1} \SpecialCharTok{|}\NormalTok{ subject),}
        \AttributeTok{family =} \FunctionTok{poisson}\NormalTok{()}
\NormalTok{        ) }\SpecialCharTok{+}
    \FunctionTok{bf}\NormalTok{(}
\NormalTok{        letter\_size }\SpecialCharTok{\textasciitilde{}} \DecValTok{1} \SpecialCharTok{+}\NormalTok{ group }\SpecialCharTok{*}\NormalTok{ frequency }\SpecialCharTok{*}\NormalTok{ grapheme\_complexity }\SpecialCharTok{*}
\NormalTok{            graphomotor\_difficulty }\SpecialCharTok{+}\NormalTok{ (}\DecValTok{1} \SpecialCharTok{|}\NormalTok{ subject),}
        \AttributeTok{family =} \FunctionTok{shifted\_lognormal}\NormalTok{()}
\NormalTok{        )}

\CommentTok{\# defining the priors for the multilevel generalised model}
\NormalTok{priors\_generalised }\OtherTok{\textless{}{-}} \FunctionTok{c}\NormalTok{(}
    \FunctionTok{prior}\NormalTok{(}\FunctionTok{normal}\NormalTok{(}\DecValTok{1}\NormalTok{, }\FloatTok{0.5}\NormalTok{), }\AttributeTok{class =}\NormalTok{ Intercept, }\AttributeTok{resp =} \StringTok{"duration"}\NormalTok{),}
    \FunctionTok{prior}\NormalTok{(}\FunctionTok{normal}\NormalTok{(}\DecValTok{0}\NormalTok{, }\FloatTok{0.5}\NormalTok{), }\AttributeTok{class =}\NormalTok{ b, }\AttributeTok{resp =} \StringTok{"duration"}\NormalTok{),}
    \FunctionTok{prior}\NormalTok{(}\FunctionTok{exponential}\NormalTok{(}\FloatTok{0.1}\NormalTok{), }\AttributeTok{class =}\NormalTok{ sd, }\AttributeTok{resp =} \StringTok{"duration"}\NormalTok{),}
    \FunctionTok{prior}\NormalTok{(}\FunctionTok{exponential}\NormalTok{(}\FloatTok{0.1}\NormalTok{), }\AttributeTok{class =}\NormalTok{ sigma, }\AttributeTok{resp =} \StringTok{"duration"}\NormalTok{),}
    \FunctionTok{prior}\NormalTok{(}\FunctionTok{normal}\NormalTok{(}\DecValTok{2}\NormalTok{, }\FloatTok{0.5}\NormalTok{), }\AttributeTok{class =}\NormalTok{ Intercept, }\AttributeTok{resp =} \StringTok{"meanvelocity"}\NormalTok{),}
    \FunctionTok{prior}\NormalTok{(}\FunctionTok{normal}\NormalTok{(}\DecValTok{0}\NormalTok{, }\FloatTok{0.5}\NormalTok{), }\AttributeTok{class =}\NormalTok{ b, }\AttributeTok{resp =} \StringTok{"meanvelocity"}\NormalTok{),}
    \FunctionTok{prior}\NormalTok{(}\FunctionTok{exponential}\NormalTok{(}\FloatTok{0.1}\NormalTok{), }\AttributeTok{class =}\NormalTok{ sd, }\AttributeTok{resp =} \StringTok{"meanvelocity"}\NormalTok{),}
    \FunctionTok{prior}\NormalTok{(}\FunctionTok{exponential}\NormalTok{(}\FloatTok{0.1}\NormalTok{), }\AttributeTok{class =}\NormalTok{ sigma, }\AttributeTok{resp =} \StringTok{"meanvelocity"}\NormalTok{),}
    \FunctionTok{prior}\NormalTok{(}\FunctionTok{normal}\NormalTok{(}\DecValTok{1}\NormalTok{, }\FloatTok{0.5}\NormalTok{), }\AttributeTok{class =}\NormalTok{ Intercept, }\AttributeTok{resp =} \StringTok{"numberofstops"}\NormalTok{),}
    \FunctionTok{prior}\NormalTok{(}\FunctionTok{normal}\NormalTok{(}\DecValTok{0}\NormalTok{, }\FloatTok{0.5}\NormalTok{), }\AttributeTok{class =}\NormalTok{ b, }\AttributeTok{resp =} \StringTok{"numberofstops"}\NormalTok{),}
    \FunctionTok{prior}\NormalTok{(}\FunctionTok{exponential}\NormalTok{(}\FloatTok{0.1}\NormalTok{), }\AttributeTok{class =}\NormalTok{ sd, }\AttributeTok{resp =} \StringTok{"numberofstops"}\NormalTok{),}
    \FunctionTok{prior}\NormalTok{(}\FunctionTok{normal}\NormalTok{(}\DecValTok{2}\NormalTok{, }\FloatTok{0.5}\NormalTok{), }\AttributeTok{class =}\NormalTok{ Intercept, }\AttributeTok{resp =} \StringTok{"lettersize"}\NormalTok{),}
    \FunctionTok{prior}\NormalTok{(}\FunctionTok{normal}\NormalTok{(}\DecValTok{0}\NormalTok{, }\FloatTok{0.5}\NormalTok{), }\AttributeTok{class =}\NormalTok{ b, }\AttributeTok{resp =} \StringTok{"lettersize"}\NormalTok{),}
    \FunctionTok{prior}\NormalTok{(}\FunctionTok{exponential}\NormalTok{(}\FloatTok{0.1}\NormalTok{), }\AttributeTok{class =}\NormalTok{ sd, }\AttributeTok{resp =} \StringTok{"lettersize"}\NormalTok{),}
    \FunctionTok{prior}\NormalTok{(}\FunctionTok{exponential}\NormalTok{(}\FloatTok{0.1}\NormalTok{), }\AttributeTok{class =}\NormalTok{ sigma, }\AttributeTok{resp =} \StringTok{"lettersize"}\NormalTok{)}
\NormalTok{    )}
\end{Highlighting}
\end{Shaded}

\begin{Shaded}
\begin{Highlighting}[]
\CommentTok{\# centering and reordering predictors}
\NormalTok{df2 }\OtherTok{\textless{}{-}}\NormalTok{ df }\SpecialCharTok{\%\textgreater{}\%}
    \FunctionTok{mutate}\NormalTok{(}
        \AttributeTok{group =} \FunctionTok{factor}\NormalTok{(}
            \AttributeTok{x =}\NormalTok{ group,}
            \AttributeTok{levels =} \FunctionTok{c}\NormalTok{(}\StringTok{"CP"}\NormalTok{, }\StringTok{"CE"}\NormalTok{, }\StringTok{"CM"}\NormalTok{),}
            \AttributeTok{labels =} \FunctionTok{c}\NormalTok{(}\StringTok{"Grade1"}\NormalTok{, }\StringTok{"Grade3"}\NormalTok{, }\StringTok{"Grade5"}\NormalTok{)}
\NormalTok{            ),}
        \AttributeTok{frequency =} \FunctionTok{factor}\NormalTok{(}
            \AttributeTok{x =}\NormalTok{ frequency,}
            \AttributeTok{levels =} \FunctionTok{c}\NormalTok{(}\StringTok{"LF"}\NormalTok{, }\StringTok{"HF"}\NormalTok{),}
            \AttributeTok{labels =} \FunctionTok{c}\NormalTok{(}\StringTok{"LF"}\NormalTok{, }\StringTok{"HF"}\NormalTok{)}
\NormalTok{            ),}
        \AttributeTok{grapheme\_complexity =} \FunctionTok{factor}\NormalTok{(}
            \AttributeTok{x =}\NormalTok{ grapheme\_complexity,}
            \AttributeTok{levels =} \FunctionTok{c}\NormalTok{(}\StringTok{"Simple"}\NormalTok{, }\StringTok{"Complex"}\NormalTok{),}
            \AttributeTok{labels =} \FunctionTok{c}\NormalTok{(}\StringTok{"Simple"}\NormalTok{, }\StringTok{"Complex"}\NormalTok{)}
\NormalTok{            ),}
        \AttributeTok{graphomotor\_difficulty =} \FunctionTok{factor}\NormalTok{(}
            \AttributeTok{x =}\NormalTok{ graphomotor\_difficulty,}
            \AttributeTok{levels =} \FunctionTok{c}\NormalTok{(}\StringTok{"EL"}\NormalTok{, }\StringTok{"HL"}\NormalTok{),}
            \AttributeTok{labels =} \FunctionTok{c}\NormalTok{(}\StringTok{"t"}\NormalTok{, }\StringTok{"f"}\NormalTok{)}
\NormalTok{            )}
\NormalTok{        ) }\SpecialCharTok{\%\textgreater{}\%}
    \CommentTok{\# removes rows where duration is equal to 0}
    \FunctionTok{filter}\NormalTok{(duration }\SpecialCharTok{!=} \DecValTok{0}\NormalTok{)}

\CommentTok{\# defining contrasts}
\FunctionTok{contrasts}\NormalTok{(df2}\SpecialCharTok{$}\NormalTok{frequency) }\OtherTok{\textless{}{-}} \FunctionTok{c}\NormalTok{(}\SpecialCharTok{{-}}\FloatTok{0.5}\NormalTok{, }\SpecialCharTok{+}\FloatTok{0.5}\NormalTok{)}
\FunctionTok{contrasts}\NormalTok{(df2}\SpecialCharTok{$}\NormalTok{grapheme\_complexity) }\OtherTok{\textless{}{-}} \FunctionTok{c}\NormalTok{(}\SpecialCharTok{{-}}\FloatTok{0.5}\NormalTok{, }\SpecialCharTok{+}\FloatTok{0.5}\NormalTok{)}
\FunctionTok{contrasts}\NormalTok{(df2}\SpecialCharTok{$}\NormalTok{graphomotor\_difficulty) }\OtherTok{\textless{}{-}} \FunctionTok{c}\NormalTok{(}\SpecialCharTok{{-}}\FloatTok{0.5}\NormalTok{, }\SpecialCharTok{+}\FloatTok{0.5}\NormalTok{)}

\CommentTok{\# fitting the model}
\NormalTok{mod2 }\OtherTok{\textless{}{-}} \FunctionTok{brm}\NormalTok{(}
    \AttributeTok{formula =}\NormalTok{ formula\_generalised }\SpecialCharTok{+} \FunctionTok{set\_rescor}\NormalTok{(}\AttributeTok{rescor =} \ConstantTok{FALSE}\NormalTok{),}
    \AttributeTok{prior =}\NormalTok{ priors\_generalised,}
    \AttributeTok{chains =} \DecValTok{4}\NormalTok{, }\AttributeTok{cores =} \DecValTok{4}\NormalTok{,}
    \AttributeTok{warmup =} \DecValTok{2000}\NormalTok{, }\AttributeTok{iter =} \FloatTok{1e4}\NormalTok{,}
    \AttributeTok{control =} \FunctionTok{list}\NormalTok{(}\AttributeTok{adapt\_delta =} \FloatTok{0.95}\NormalTok{),}
    \AttributeTok{data =}\NormalTok{ df2,}
    \AttributeTok{sample\_prior =} \ConstantTok{TRUE}\NormalTok{,}
    \AttributeTok{file =} \StringTok{"models/multilevel\_generalised\_model"}
\NormalTok{    )}
\end{Highlighting}
\end{Shaded}

We then fit this model below using the \texttt{brms::brm()} function. We run four chains, each for 10000 iterations and using the first 2000 iterations used as warmup (i.e., the first 2000 samples of each chain are discarded from the final analysis). This results in a total of \(4 \times (10000 - 2000) = 32000\) samples from the (joint) posterior distribution that will be used for inference.

\hypertarget{evaluating-the-model}{%
\subsection{Evaluating the model}\label{evaluating-the-model}}

One way of evaluating the model is to evaluate its predictions. In Figure \ref{fig:ppc}, we depict the distribution of the raw data along with the distribution of 100 simulated datasets (a posterior predictive check, as introduced previously).

\begin{figure}[!htb]

{\centering \includegraphics[width=1\linewidth]{supplementary_materials_files/figure-latex/ppc-1} 

}

\caption{Posterior predictive checking. The dark blue line represents the distribution of raw data whereas light blue lines represent data simulated from the posterior distribution.}\label{fig:ppc}
\end{figure}

As can be seen from Figure \ref{fig:ppc}, the model seems pretty good at simulating data that looks like the observed data. From this predictive/sampling distribution (i.e., the distribution of simulated data sets), so-called ``Bayesian \emph{p}-values'' can be computed to quantify the compatibility between the observed data and the proposed model.

\newpage

\hypertarget{hypothesis-testing}{%
\subsection{Hypothesis testing}\label{hypothesis-testing}}

We can test any arbitrary hypothesis using the \texttt{brms::hypothesis()} method, which is computing a Bayes factor via the Savage-Dickey method (\protect\hyperlink{ref-wagenmakers_bayesian_2010}{Wagenmakers, Lodewyckx, Kuriyal, \& Grasman, 2010}). This method consists in comparing the posterior probability density to the prior probability density for some hypothesised value for the parameter of interest (e.g., \(\theta = 0\)). For instance, we test below the hypothesis according to which the effect of graphemic complexity in Grade 1 would be null.

\begin{Shaded}
\begin{Highlighting}[]
\CommentTok{\# testing whether the effect of grapheme complexity on duration equal to 0}
\NormalTok{hyp }\OtherTok{\textless{}{-}} \FunctionTok{hypothesis}\NormalTok{(}\AttributeTok{x =}\NormalTok{ mod2, }\AttributeTok{hypothesis =} \StringTok{"duration\_grapheme\_complexity1 = 0"}\NormalTok{)}

\CommentTok{\# prints the output}
\FunctionTok{print}\NormalTok{(hyp)}
\end{Highlighting}
\end{Shaded}

\begin{verbatim}
## Hypothesis Tests for class b:
##                 Hypothesis Estimate Est.Error CI.Lower CI.Upper Evid.Ratio
## 1 (duration_graphem... = 0        0      0.05    -0.11      0.1       9.87
##   Post.Prob Star
## 1      0.91     
## ---
## 'CI': 90%-CI for one-sided and 95%-CI for two-sided hypotheses.
## '*': For one-sided hypotheses, the posterior probability exceeds 95%;
## for two-sided hypotheses, the value tested against lies outside the 95%-CI.
## Posterior probabilities of point hypotheses assume equal prior probabilities.
\end{verbatim}

\begin{Shaded}
\begin{Highlighting}[]
\CommentTok{\# plotting it}
\FunctionTok{data.frame}\NormalTok{(}\AttributeTok{posterior =}\NormalTok{ hyp}\SpecialCharTok{$}\NormalTok{samples}\SpecialCharTok{$}\NormalTok{H1, }\AttributeTok{prior =}\NormalTok{ hyp}\SpecialCharTok{$}\NormalTok{prior\_samples}\SpecialCharTok{$}\NormalTok{H1) }\SpecialCharTok{\%\textgreater{}\%}
  \FunctionTok{gather}\NormalTok{(type, value) }\SpecialCharTok{\%\textgreater{}\%}
  \FunctionTok{ggplot}\NormalTok{(}\FunctionTok{aes}\NormalTok{(}\AttributeTok{x =}\NormalTok{ value, }\AttributeTok{fill =}\NormalTok{ type) ) }\SpecialCharTok{+}
  \FunctionTok{geom\_vline}\NormalTok{(}\AttributeTok{xintercept =} \DecValTok{0}\NormalTok{, }\AttributeTok{linetype =} \DecValTok{2}\NormalTok{, }\AttributeTok{alpha =} \DecValTok{1}\NormalTok{) }\SpecialCharTok{+}
  \FunctionTok{geom\_area}\NormalTok{(}\AttributeTok{stat =} \StringTok{"density"}\NormalTok{, }\AttributeTok{alpha =} \FloatTok{0.8}\NormalTok{, }\AttributeTok{position =} \StringTok{"identity"}\NormalTok{) }\SpecialCharTok{+}
  \FunctionTok{theme\_bw}\NormalTok{(}\AttributeTok{base\_size =} \DecValTok{12}\NormalTok{, }\AttributeTok{base\_family =} \StringTok{"Open Sans"}\NormalTok{) }\SpecialCharTok{+}
  \FunctionTok{labs}\NormalTok{(}\AttributeTok{x =} \FunctionTok{expression}\NormalTok{(beta[grapheme\_complexity]), }\AttributeTok{y =} \StringTok{"Probability density"}\NormalTok{) }\SpecialCharTok{+}
  \FunctionTok{scale\_fill\_brewer}\NormalTok{(}\AttributeTok{palette =} \StringTok{"Dark2"}\NormalTok{) }\SpecialCharTok{+}
  \FunctionTok{theme}\NormalTok{(}\AttributeTok{legend.title =} \FunctionTok{element\_blank}\NormalTok{() ) }\SpecialCharTok{+}
  \FunctionTok{coord\_cartesian}\NormalTok{(}\AttributeTok{xlim =} \FunctionTok{c}\NormalTok{(}\SpecialCharTok{{-}}\DecValTok{2}\NormalTok{, }\DecValTok{2}\NormalTok{) )}
\end{Highlighting}
\end{Shaded}

\begin{figure}[!htb]

{\centering \includegraphics[width=0.75\linewidth]{supplementary_materials_files/figure-latex/hypothesis1-1} 

}

\caption{Hypothesis testing via the Savage-Dickey method. The resulting Bayes factor (BF) is the ratio of the height (i.e., the density probability) of the posterior versus prior distribution at some value of interest for the parameter (here it is 0).}\label{fig:hypothesis1}
\end{figure}

The resulting Bayes factor (called ``Evid. Ratio'' in the output) may be interpreted as follows: the observed data are 9.87 more likely under the hypothesis of null effect than under the hypothesis of a non-null effect. From the BF in favour of the null hypothesis (relative to the alternative hypothesis), we can compute the BF in favour of the alternative hypothesis (relative to the null hypothesis), using \(\text{BF}_{10} = 1 / \text{BF}_{01}\) (we report the \(\text{BF}_{10}\) in the following). Alternatively, the BF can be interpreted as an \emph{updating factor}, indicating by ``how much'' we should update our \emph{prior odds} (the ratio of the a priori probability of \(\mathcal{H}_{0}\) versus \(\mathcal{H}_{1}\)) to convert them into \emph{posterior odds} (the ratio of the a posteriori probability of \(\mathcal{H}_{0}\) versus \(\mathcal{H}_{1}\)).

\newpage

\hypertarget{interpretation-of-the-results-for-each-variable}{%
\section{Interpretation of the results for each variable}\label{interpretation-of-the-results-for-each-variable}}

Now that we have fitted the model, we are left with the task of interpreting the output from the model. The output of the model is a (joint) posterior distribution over all parameters of the model. We can marginalise this joint distribution to obtain the (marginal) posterior distribution on each parameter. To summarise this distribution, we can retrieve samples from the joint posterior distribution.

\begin{Shaded}
\begin{Highlighting}[]
\CommentTok{\# retrieves posterior samples (for all parameters)}
\NormalTok{posterior\_samples }\OtherTok{\textless{}{-}} \FunctionTok{as\_draws\_df}\NormalTok{(mod2)}

\CommentTok{\# displays a summary}
\NormalTok{posterior\_summary }\OtherTok{\textless{}{-}} \FunctionTok{summarise\_draws}\NormalTok{(posterior\_samples)}

\CommentTok{\# displays the first six rows}
\FunctionTok{head}\NormalTok{(posterior\_summary)}
\end{Highlighting}
\end{Shaded}

\begin{verbatim}
## # A tibble: 6 x 10
##   variable       mean median     sd    mad     q5    q95  rhat ess_bulk ess_tail
##   <chr>         <dbl>  <dbl>  <dbl>  <dbl>  <dbl>  <dbl> <dbl>    <dbl>    <dbl>
## 1 b_duration_~  1.08   1.08  0.0310 0.0304  1.03   1.13   1.00   29023.   25012.
## 2 b_meanveloc~  2.34   2.34  0.0271 0.0263  2.29   2.38   1.00   30342.   20075.
## 3 b_numberofs~  1.04   1.04  0.0367 0.0354  0.977  1.10   1.00   41575.   26004.
## 4 b_lettersiz~  2.29   2.29  0.0234 0.0229  2.25   2.33   1.00   32779.   23505.
## 5 b_duration_~ -0.866 -0.865 0.0386 0.0384 -0.931 -0.804  1.00   23574.   24590.
## 6 b_duration_~ -0.967 -0.966 0.0414 0.0418 -1.04  -0.900  1.00   20887.   23486.
\end{verbatim}

The above command outputs a matrix with parameters of the model in columns and posterior samples in rows. Let's examine these results for each parameter in more details. For instance, Figure \ref{fig:posterior-intercept-drift} represents the posterior distribution of the average letter duration in Grade-1 children.

\begin{Shaded}
\begin{Highlighting}[]
\CommentTok{\# retrieves the posterior samples for the average letter duration in Grade 1}
\NormalTok{average\_duration\_grade1 }\OtherTok{\textless{}{-}}\NormalTok{ posterior\_samples}\SpecialCharTok{$}\NormalTok{b\_duration\_Intercept }\SpecialCharTok{+}
\NormalTok{    posterior\_samples}\SpecialCharTok{$}\NormalTok{ndt\_duration}

\CommentTok{\# plotting it}
\FunctionTok{plotPost}\NormalTok{(}
  \AttributeTok{paramSampleVec =} \FunctionTok{exp}\NormalTok{(average\_duration\_grade1), }\AttributeTok{showMode =} \ConstantTok{TRUE}\NormalTok{,}
  \AttributeTok{xlab =} \FunctionTok{expression}\NormalTok{(}\FunctionTok{paste}\NormalTok{(alpha[duration][}\FunctionTok{paste}\NormalTok{(}\StringTok{"["}\NormalTok{, Grade1, }\StringTok{"]"}\NormalTok{)] ) )}
\NormalTok{  )}
\end{Highlighting}
\end{Shaded}

\begin{figure}[!htb]

{\centering \includegraphics[width=0.75\linewidth]{supplementary_materials_files/figure-latex/posterior-intercept-drift-1} 

}

\caption{Posterior distribution of the intercept (i.e., the average letter duration in Grade 1). The mode (i.e., the most probable value) and the 95\% credible (highest density) interval are also displayed.}\label{fig:posterior-intercept-drift}
\end{figure}

Recall that we used a logarithmic link function, therefore the median letter duration is given by \(\exp(\alpha + \text{shift})\).

\hypertarget{letter-duration}{%
\subsection{Letter duration}\label{letter-duration}}

Table \ref{tab:duration-summary} reports the estimates (median of the posterior distribution) and associated 95\% credible intervals and \(\text{BF}\)s for all parameters regarding the letter duration variable.

\begin{lltable}

\begin{TableNotes}[para]
\normalsize{\textit{Note.} For each slope (for each line), the first two columns represent the
    estimated most probable value and its standard error (SE). The 'Lower' and
    'Upper' columns contain the lower and upper bounds of the 95\% CrI, whereas
    the 'Rhat' column reports the Gelman-Rubin statistic. The last column reports
    the Bayes factor in favour of the alternative hypothesis, relative to the
    null hypothesis (BF10).}
\end{TableNotes}

\scriptsize{

\begin{longtable}{ccccccc}\noalign{\getlongtablewidth\global\LTcapwidth=\longtablewidth}
\caption{\label{tab:duration-summary}Estimates and BFs for the slopes for letter duration.}\\
\toprule
Term & \multicolumn{1}{c}{Estimate} & \multicolumn{1}{c}{MAD} & \multicolumn{1}{c}{Lower} & \multicolumn{1}{c}{Upper} & \multicolumn{1}{c}{Rhat} & \multicolumn{1}{c}{BF10}\\
\midrule
\endfirsthead
\caption*{\normalfont{Table \ref{tab:duration-summary} continued}}\\
\toprule
Term & \multicolumn{1}{c}{Estimate} & \multicolumn{1}{c}{MAD} & \multicolumn{1}{c}{Lower} & \multicolumn{1}{c}{Upper} & \multicolumn{1}{c}{Rhat} & \multicolumn{1}{c}{BF10}\\
\midrule
\endhead
Intercept & 1.083 & 0.030 & 1.020 & 1.142 & 1.000 & NA\\
groupGrade3 & -0.865 & 0.038 & -0.943 & -0.793 & 1.000 & 7.281 x 10\textasciicircum{}17\\
groupGrade5 & -0.966 & 0.042 & -1.050 & -0.889 & 1.000 & 3.586 x 10\textasciicircum{}15\\
frequency & -0.055 & 0.050 & -0.157 & 0.049 & 1.000 & 0.192\\
grapheme\_complexity & -0.005 & 0.050 & -0.106 & 0.097 & 1.000 & 0.101\\
graphomotor\_difficulty & 0.423 & 0.051 & 0.320 & 0.526 & 1.000 & 9.762 x 10\textasciicircum{}15\\
groupGrade3:frequency & 0.074 & 0.062 & -0.049 & 0.196 & 1.000 & 0.258\\
groupGrade5:frequency & 0.027 & 0.062 & -0.092 & 0.148 & 1.000 & 0.136\\
groupGrade3:grapheme\_complexity & 0.011 & 0.062 & -0.112 & 0.132 & 1.000 & 0.13\\
groupGrade5:grapheme\_complexity & 0.004 & 0.061 & -0.115 & 0.124 & 1.000 & 0.124\\
frequency:grapheme\_complexity & 0.046 & 0.097 & -0.151 & 0.244 & 1.000 & 0.225\\
groupGrade3:graphomotor\_difficulty & 0.001 & 0.063 & -0.121 & 0.127 & 1.000 & 0.132\\
groupGrade5:graphomotor\_difficulty & 0.099 & 0.064 & -0.024 & 0.225 & 1.000 & 0.436\\
frequency:graphomotor\_difficulty & -0.071 & 0.097 & -0.268 & 0.128 & 1.000 & 0.261\\
grapheme\_complexity:graphomotor\_difficulty & -0.008 & 0.098 & -0.203 & 0.188 & 1.000 & 0.197\\
groupGrade3:frequency:grapheme\_complexity & 0.010 & 0.120 & -0.229 & 0.248 & 1.000 & 0.249\\
groupGrade5:frequency:grapheme\_complexity & -0.101 & 0.119 & -0.337 & 0.131 & 1.000 & 0.352\\
groupGrade3:frequency:graphomotor\_difficulty & 0.007 & 0.122 & -0.233 & 0.243 & 1.000 & 0.246\\
groupGrade5:frequency:graphomotor\_difficulty & 0.068 & 0.120 & -0.167 & 0.308 & 1.000 & 0.285\\
groupGrade3:grapheme\_complexity:graphomotor\_difficulty & 0.072 & 0.120 & -0.162 & 0.309 & 1.000 & 0.291\\
groupGrade5:grapheme\_complexity:graphomotor\_difficulty & 0.035 & 0.119 & -0.201 & 0.270 & 1.000 & 0.252\\
frequency:grapheme\_complexity:graphomotor\_difficulty & -0.040 & 0.176 & -0.401 & 0.321 & 1.000 & 0.368\\
groupGrade3:frequency:grapheme\_complexity:graphomotor\_difficulty & -0.006 & 0.217 & -0.434 & 0.424 & 1.000 & 0.442\\
groupGrade5:frequency:grapheme\_complexity:graphomotor\_difficulty & 0.134 & 0.213 & -0.283 & 0.548 & 1.000 & 0.529\\
\bottomrule
\addlinespace
\insertTableNotes
\end{longtable}

}

\end{lltable}

These estimations are better understood visually. Thus, we plot the predictions of this model against raw data in Figure \ref{fig:duration-predictions-plot}.

\begin{Shaded}
\begin{Highlighting}[]
\CommentTok{\# retrieving the model\textquotesingle{}s predictions}
\NormalTok{duration\_predictions }\OtherTok{\textless{}{-}}\NormalTok{ df2 }\SpecialCharTok{\%\textgreater{}\%}
    \FunctionTok{data\_grid}\NormalTok{(graphomotor\_difficulty, grapheme\_complexity, frequency, group) }\SpecialCharTok{\%\textgreater{}\%}
    \FunctionTok{cbind}\NormalTok{(., }\FunctionTok{fitted}\NormalTok{(}
        \AttributeTok{object =}\NormalTok{ mod2, }\AttributeTok{newdata =}\NormalTok{ ., }\AttributeTok{resp =} \StringTok{"duration"}\NormalTok{,}
        \AttributeTok{scale =} \StringTok{"response"}\NormalTok{, }\AttributeTok{probs =} \FunctionTok{c}\NormalTok{(}\FloatTok{0.025}\NormalTok{, }\FloatTok{0.975}\NormalTok{),}
        \AttributeTok{re\_formula =} \ConstantTok{NA}\NormalTok{, }\AttributeTok{robust =} \ConstantTok{TRUE}
\NormalTok{        ) ) }\SpecialCharTok{\%\textgreater{}\%}
\NormalTok{    ungroup }\SpecialCharTok{\%\textgreater{}\%}
\NormalTok{    dplyr}\SpecialCharTok{::}\FunctionTok{rename}\NormalTok{(}\AttributeTok{estimate =}\NormalTok{ Estimate, }\AttributeTok{mad =}\NormalTok{ Est.Error, }\AttributeTok{lower =}\NormalTok{ Q2}\FloatTok{.5}\NormalTok{, }\AttributeTok{upper =}\NormalTok{ Q97}\FloatTok{.5}\NormalTok{)}
\end{Highlighting}
\end{Shaded}

\begin{figure}[!htb]

{\centering \includegraphics[width=1\linewidth]{supplementary_materials_files/figure-latex/duration-predictions-plot-1} 

}

\caption{Letter duration by grade and word frequency (x-axis), grapheme complexity (in colour), and graphomotor difficulty (in panels). Transparent points represent individual data per participant. The surimposed dots and intervals represent the model's predictions (median and 95\% credible interval of the posterior distribution).}\label{fig:duration-predictions-plot}
\end{figure}

As can be seen in Figure \ref{fig:duration-predictions-plot}, the model predicts larger letter duration for hard letters as compared to easy letters for each grade. As can be seen from Table \ref{tab:duration-summary}, the only BFs favouring the alternative hypothesis (relative to the null hypothesis) are the BFs for the difference between Grade 1 and Grade 3 in average letter duration (\(\beta\) = -0.865, 95\% CrI {[}-0.943, -0.793{]}, BF\textsubscript{10} = 7.281 x 10\^{}17), as well as the difference between Grade 1 and Grade 5 (\(\beta\) = -0.966, 95\% CrI {[}-1.05, -0.889{]}, BF\textsubscript{10} = 3.586 x 10\^{}15), and the effect of graphomotor difficulty in Grade 1 (\(\beta\) = 0.423, 95\% CrI {[}0.32, 0.526{]}, BF\textsubscript{10} = 9.762 x 10\^{}15). Predictions from this model for each condition are also summarised in Table \ref{tab:duration-predictions-summary}.

\begin{table}[htb]

\begin{center}
\begin{threeparttable}

\caption{\label{tab:duration-predictions-summary}Estimated letter duration in each condition.}

\scriptsize{

\begin{tabular}{cccccccc}
\toprule
Group & \multicolumn{1}{c}{Frequency} & \multicolumn{1}{c}{Grapheme complexity} & \multicolumn{1}{c}{Graphomotor difficulty} & \multicolumn{1}{c}{Estimate} & \multicolumn{1}{c}{MAD} & \multicolumn{1}{c}{Lower} & \multicolumn{1}{c}{Upper}\\
\midrule
Grade1 & LF & Simple & t & 2.768 & 0.186 & 2.416 & 3.172\\
Grade1 & LF & Simple & f & 4.296 & 0.287 & 3.745 & 4.923\\
Grade1 & LF & Complex & t & 2.680 & 0.179 & 2.334 & 3.077\\
Grade1 & LF & Complex & f & 4.203 & 0.291 & 3.651 & 4.846\\
Grade1 & HF & Simple & t & 2.632 & 0.176 & 2.303 & 3.011\\
Grade1 & HF & Simple & f & 3.883 & 0.263 & 3.387 & 4.458\\
Grade1 & HF & Complex & t & 2.713 & 0.182 & 2.367 & 3.122\\
Grade1 & HF & Complex & f & 3.899 & 0.261 & 3.401 & 4.469\\
Grade3 & LF & Simple & t & 1.196 & 0.084 & 1.040 & 1.381\\
Grade3 & LF & Simple & f & 1.760 & 0.127 & 1.522 & 2.040\\
Grade3 & LF & Complex & t & 1.127 & 0.080 & 0.980 & 1.304\\
Grade3 & LF & Complex & f & 1.797 & 0.127 & 1.556 & 2.074\\
Grade3 & HF & Simple & t & 1.211 & 0.085 & 1.053 & 1.399\\
Grade3 & HF & Simple & f & 1.710 & 0.121 & 1.482 & 1.980\\
Grade3 & HF & Complex & t & 1.226 & 0.087 & 1.063 & 1.418\\
Grade3 & HF & Complex & f & 1.802 & 0.129 & 1.561 & 2.090\\
Grade5 & LF & Simple & t & 1.033 & 0.070 & 0.902 & 1.185\\
Grade5 & LF & Simple & f & 1.693 & 0.120 & 1.470 & 1.953\\
Grade5 & LF & Complex & t & 1.068 & 0.072 & 0.931 & 1.230\\
Grade5 & LF & Complex & f & 1.722 & 0.121 & 1.493 & 1.985\\
Grade5 & HF & Simple & t & 1.057 & 0.072 & 0.923 & 1.213\\
Grade5 & HF & Simple & f & 1.654 & 0.115 & 1.435 & 1.909\\
Grade5 & HF & Complex & t & 0.996 & 0.066 & 0.869 & 1.142\\
Grade5 & HF & Complex & f & 1.667 & 0.117 & 1.447 & 1.924\\
\bottomrule
\addlinespace
\end{tabular}

}

\begin{tablenotes}[para]
\normalsize{\textit{Note.} For each condition, the 'Estimate' and 'MAD' columns contain the
    median and the median absolute deviation (MAD) of the posterior distribution,
    respectively. The 'Lower' and 'Upper' columns contain the lower and upper
    bounds of the 95\% credible interval.}
\end{tablenotes}

\end{threeparttable}
\end{center}

\end{table}

\hypertarget{number-of-stops}{%
\subsection{Number of stops}\label{number-of-stops}}

Table \ref{tab:stops-summary} reports the estimates (median of the posterior distribution) and associated 95\% credible intervals and \(\text{BF}\)s for all parameters regarding the number of stops.

\begin{lltable}

\begin{TableNotes}[para]
\normalsize{\textit{Note.} For each slope (for each line), the first two columns represent the
    estimated most probable value and its standard error (SE). The 'Lower' and
    'Upper' columns contain the lower and upper bounds of the 95\% CrI, whereas
    the 'Rhat' column reports the Gelman-Rubin statistic. The last column reports
    the Bayes factor in favour of the alternative hypothesis, relative to the
    null hypothesis (BF10).}
\end{TableNotes}

\scriptsize{

\begin{longtable}{ccccccc}\noalign{\getlongtablewidth\global\LTcapwidth=\longtablewidth}
\caption{\label{tab:stops-summary}Estimates and BFs for the slopes for the number of stops.}\\
\toprule
Term & \multicolumn{1}{c}{Estimate} & \multicolumn{1}{c}{MAD} & \multicolumn{1}{c}{Lower} & \multicolumn{1}{c}{Upper} & \multicolumn{1}{c}{Rhat} & \multicolumn{1}{c}{BF10}\\
\midrule
\endfirsthead
\caption*{\normalfont{Table \ref{tab:stops-summary} continued}}\\
\toprule
Term & \multicolumn{1}{c}{Estimate} & \multicolumn{1}{c}{MAD} & \multicolumn{1}{c}{Lower} & \multicolumn{1}{c}{Upper} & \multicolumn{1}{c}{Rhat} & \multicolumn{1}{c}{BF10}\\
\midrule
\endhead
Intercept & 1.037 & 0.035 & 0.965 & 1.109 & 1.000 & NA\\
groupGrade3 & -1.384 & 0.073 & -1.529 & -1.243 & 1.000 & 4.244 x 10\textasciicircum{}16\\
groupGrade5 & -1.556 & 0.077 & -1.709 & -1.409 & 1.000 & 6.510 x 10\textasciicircum{}15\\
frequency & -0.180 & 0.070 & -0.324 & -0.034 & 1.000 & 2.746\\
grapheme\_complexity & 0.031 & 0.070 & -0.111 & 0.175 & 1.000 & 0.155\\
graphomotor\_difficulty & -0.051 & 0.070 & -0.192 & 0.094 & 1.000 & 0.184\\
groupGrade3:frequency & 0.292 & 0.140 & 0.019 & 0.568 & 1.000 & 2.537\\
groupGrade5:frequency & 0.128 & 0.147 & -0.161 & 0.409 & 1.000 & 0.417\\
groupGrade3:grapheme\_complexity & 0.051 & 0.142 & -0.227 & 0.329 & 1.000 & 0.298\\
groupGrade5:grapheme\_complexity & -0.069 & 0.149 & -0.356 & 0.220 & 1.000 & 0.334\\
frequency:grapheme\_complexity & 0.177 & 0.133 & -0.096 & 0.446 & 1.000 & 0.653\\
groupGrade3:graphomotor\_difficulty & 0.112 & 0.140 & -0.164 & 0.390 & 1.000 & 0.378\\
groupGrade5:graphomotor\_difficulty & 0.061 & 0.144 & -0.226 & 0.346 & 1.000 & 0.313\\
frequency:graphomotor\_difficulty & -0.312 & 0.134 & -0.581 & -0.032 & 1.000 & 3.307\\
grapheme\_complexity:graphomotor\_difficulty & -0.047 & 0.135 & -0.315 & 0.224 & 1.000 & 0.285\\
groupGrade3:frequency:grapheme\_complexity & -0.234 & 0.254 & -0.731 & 0.258 & 1.000 & 0.791\\
groupGrade5:frequency:grapheme\_complexity & -0.147 & 0.260 & -0.653 & 0.358 & 1.000 & 0.602\\
groupGrade3:frequency:graphomotor\_difficulty & 0.102 & 0.257 & -0.396 & 0.605 & 1.000 & 0.55\\
groupGrade5:frequency:graphomotor\_difficulty & 0.326 & 0.255 & -0.176 & 0.834 & 1.000 & 1.153\\
groupGrade3:grapheme\_complexity:graphomotor\_difficulty & 0.068 & 0.254 & -0.427 & 0.568 & 1.000 & 0.529\\
groupGrade5:grapheme\_complexity:graphomotor\_difficulty & 0.004 & 0.264 & -0.511 & 0.509 & 1.000 & 0.52\\
frequency:grapheme\_complexity:graphomotor\_difficulty & 0.042 & 0.231 & -0.430 & 0.518 & 1.000 & 0.464\\
groupGrade3:frequency:grapheme\_complexity:graphomotor\_difficulty & -0.057 & 0.379 & -0.801 & 0.678 & 1.000 & 0.766\\
groupGrade5:frequency:grapheme\_complexity:graphomotor\_difficulty & 0.073 & 0.384 & -0.679 & 0.836 & 1.000 & 0.782\\
\bottomrule
\addlinespace
\insertTableNotes
\end{longtable}

}

\end{lltable}

These estimations are better understood visually. Thus, we plot the predictions of this model against raw data in Figure \ref{fig:stops-predictions-plot}.

\begin{Shaded}
\begin{Highlighting}[]
\CommentTok{\# retrieving the model\textquotesingle{}s predictions}
\NormalTok{stops\_predictions }\OtherTok{\textless{}{-}}\NormalTok{ df2 }\SpecialCharTok{\%\textgreater{}\%}
    \FunctionTok{data\_grid}\NormalTok{(graphomotor\_difficulty, grapheme\_complexity, frequency, group) }\SpecialCharTok{\%\textgreater{}\%}
    \FunctionTok{cbind}\NormalTok{(., }\FunctionTok{fitted}\NormalTok{(}
        \AttributeTok{object =}\NormalTok{ mod2, }\AttributeTok{newdata =}\NormalTok{ ., }\AttributeTok{resp =} \StringTok{"numberofstops"}\NormalTok{,}
        \AttributeTok{scale =} \StringTok{"response"}\NormalTok{, }\AttributeTok{probs =} \FunctionTok{c}\NormalTok{(}\FloatTok{0.025}\NormalTok{, }\FloatTok{0.975}\NormalTok{),}
        \AttributeTok{re\_formula =} \ConstantTok{NA}\NormalTok{, }\AttributeTok{robust =} \ConstantTok{TRUE}
\NormalTok{        ) ) }\SpecialCharTok{\%\textgreater{}\%}
\NormalTok{    ungroup }\SpecialCharTok{\%\textgreater{}\%}
\NormalTok{    dplyr}\SpecialCharTok{::}\FunctionTok{rename}\NormalTok{(}\AttributeTok{estimate =}\NormalTok{ Estimate, }\AttributeTok{mad =}\NormalTok{ Est.Error, }\AttributeTok{lower =}\NormalTok{ Q2}\FloatTok{.5}\NormalTok{, }\AttributeTok{upper =}\NormalTok{ Q97}\FloatTok{.5}\NormalTok{)}
\end{Highlighting}
\end{Shaded}

\begin{figure}[!htb]

{\centering \includegraphics[width=1\linewidth]{supplementary_materials_files/figure-latex/stops-predictions-plot-1} 

}

\caption{Number of stops by grade and word frequency (x-axis), grapheme complexity (in colour), and graphomotor difficulty (in panels). Transparent points represent individual data per participant. The surimposed dots and intervals represent the model's predictions (median and 95\% credible interval of the posterior distribution).}\label{fig:stops-predictions-plot}
\end{figure}

As can be seen in Figure \ref{fig:stops-predictions-plot}, the model most predicts an interaction between the effect of the word frequency and the effect of first-letter graphomotor difficulty in Grade 1, with infrequent words leading to a greater number of stops than frequent words for hard letters (HL) more than for easy letters (EL)
(\(\beta\) = -0.312, 95\% CrI {[}-0.581, -0.032{]}, BF\textsubscript{10} = 3.307). As can be seen from Table \ref{tab:stops-summary}, others BFs favouring the alternative hypothesis (relative to the null hypothesis) are BFs for the difference between Grade 1 and Grade 3 (\(\beta\) = -1.384, 95\% CrI {[}-1.529, -1.243{]}, BF\textsubscript{10} = 4.244 x 10\^{}16), as well as between Grade 1 and Grade 5 (\(\beta\) = -1.556, 95\% CrI {[}-1.709, -1.409{]}, BF\textsubscript{10} = 6.510 x 10\^{}15), the effect of word frequency in Grade 1 (\(\beta\) = -0.18, 95\% CrI {[}-0.324, -0.034{]}, BF\textsubscript{10} = 2.746), and Grade 3 (\(\beta\) = 0.292, 95\% CrI {[}0.019, 0.568{]}, BF\textsubscript{10} = 2.537). Predictions from this model for each condition are also summarised in Table \ref{tab:stops-predictions-summary}.

\begin{table}[htb]

\begin{center}
\begin{threeparttable}

\caption{\label{tab:stops-predictions-summary}Estimated number of stops in each condition.}

\scriptsize{

\begin{tabular}{cccccccc}
\toprule
Group & \multicolumn{1}{c}{Frequency} & \multicolumn{1}{c}{Grapheme complexity} & \multicolumn{1}{c}{Graphomotor difficulty} & \multicolumn{1}{c}{Estimate} & \multicolumn{1}{c}{MAD} & \multicolumn{1}{c}{Lower} & \multicolumn{1}{c}{Upper}\\
\midrule
Grade1 & LF & Simple & t & 2.964 & 0.275 & 2.438 & 3.585\\
Grade1 & LF & Simple & f & 3.411 & 0.302 & 2.820 & 4.070\\
Grade1 & LF & Complex & t & 2.897 & 0.272 & 2.382 & 3.515\\
Grade1 & LF & Complex & f & 3.110 & 0.289 & 2.565 & 3.753\\
Grade1 & HF & Simple & t & 2.676 & 0.258 & 2.187 & 3.246\\
Grade1 & HF & Simple & f & 2.205 & 0.227 & 1.791 & 2.716\\
Grade1 & HF & Complex & t & 3.055 & 0.283 & 2.519 & 3.689\\
Grade1 & HF & Complex & f & 2.454 & 0.240 & 2.004 & 3.002\\
Grade3 & LF & Simple & t & 0.589 & 0.114 & 0.392 & 0.855\\
Grade3 & LF & Simple & f & 0.684 & 0.126 & 0.469 & 0.967\\
Grade3 & LF & Complex & t & 0.648 & 0.120 & 0.440 & 0.923\\
Grade3 & LF & Complex & f & 0.775 & 0.135 & 0.543 & 1.075\\
Grade3 & HF & Simple & t & 0.748 & 0.133 & 0.518 & 1.053\\
Grade3 & HF & Simple & f & 0.712 & 0.127 & 0.491 & 1.001\\
Grade3 & HF & Complex & t & 0.783 & 0.138 & 0.544 & 1.093\\
Grade3 & HF & Complex & f & 0.756 & 0.134 & 0.524 & 1.059\\
Grade5 & LF & Simple & t & 0.613 & 0.113 & 0.419 & 0.866\\
Grade5 & LF & Simple & f & 0.648 & 0.116 & 0.443 & 0.910\\
Grade5 & LF & Complex & t & 0.611 & 0.115 & 0.418 & 0.872\\
Grade5 & LF & Complex & f & 0.582 & 0.110 & 0.394 & 0.830\\
Grade5 & HF & Simple & t & 0.585 & 0.112 & 0.395 & 0.838\\
Grade5 & HF & Simple & f & 0.590 & 0.111 & 0.402 & 0.844\\
Grade5 & HF & Complex & t & 0.567 & 0.109 & 0.380 & 0.811\\
Grade5 & HF & Complex & f & 0.582 & 0.112 & 0.391 & 0.838\\
\bottomrule
\addlinespace
\end{tabular}

}

\begin{tablenotes}[para]
\normalsize{\textit{Note.} For each condition, the 'Estimate' and 'MAD' columns contain the
    median and the median absolute deviation (MAD) of the posterior distribution,
    respectively. The 'Lower' and 'Upper' columns contain the lower and upper
    bounds of the 95\% credible interval.}
\end{tablenotes}

\end{threeparttable}
\end{center}

\end{table}

\hypertarget{mean-velocity}{%
\subsection{Mean velocity}\label{mean-velocity}}

Table \ref{tab:velocity-summary} reports the estimates (median of the posterior distribution) and associated 95\% credible intervals and \(\text{BF}\)s for all parameters regarding the mean velocity.

\begin{lltable}

\begin{TableNotes}[para]
\normalsize{\textit{Note.} For each slope (for each line), the first two columns represent the
    estimated most probable value and its standard error (SE). The 'Lower' and
    'Upper' columns contain the lower and upper bounds of the 95\% CrI, whereas
    the 'Rhat' column reports the Gelman-Rubin statistic. The last column reports
    the Bayes factor in favour of the alternative hypothesis, relative to the
    null hypothesis (BF10).}
\end{TableNotes}

\scriptsize{

\begin{longtable}{ccccccc}\noalign{\getlongtablewidth\global\LTcapwidth=\longtablewidth}
\caption{\label{tab:velocity-summary}Estimates and BFs for the slopes for the mean velocity.}\\
\toprule
Term & \multicolumn{1}{c}{Estimate} & \multicolumn{1}{c}{MAD} & \multicolumn{1}{c}{Lower} & \multicolumn{1}{c}{Upper} & \multicolumn{1}{c}{Rhat} & \multicolumn{1}{c}{BF10}\\
\midrule
\endfirsthead
\caption*{\normalfont{Table \ref{tab:velocity-summary} continued}}\\
\toprule
Term & \multicolumn{1}{c}{Estimate} & \multicolumn{1}{c}{MAD} & \multicolumn{1}{c}{Lower} & \multicolumn{1}{c}{Upper} & \multicolumn{1}{c}{Rhat} & \multicolumn{1}{c}{BF10}\\
\midrule
\endhead
Intercept & 2.338 & 0.026 & 2.281 & 2.387 & 1.000 & NA\\
groupGrade3 & 0.562 & 0.033 & 0.496 & 0.628 & 1.000 & 1.213 x 10\textasciicircum{}15\\
groupGrade5 & 0.793 & 0.033 & 0.729 & 0.859 & 1.000 & 1.859 x 10\textasciicircum{}16\\
frequency & 0.049 & 0.046 & -0.043 & 0.141 & 1.000 & 0.162\\
grapheme\_complexity & 0.036 & 0.047 & -0.057 & 0.130 & 1.000 & 0.123\\
graphomotor\_difficulty & 0.159 & 0.046 & 0.066 & 0.252 & 1.000 & 16.037\\
groupGrade3:frequency & -0.063 & 0.065 & -0.188 & 0.064 & 1.000 & 0.199\\
groupGrade5:frequency & -0.031 & 0.064 & -0.156 & 0.095 & 1.000 & 0.142\\
groupGrade3:grapheme\_complexity & -0.049 & 0.066 & -0.177 & 0.080 & 1.000 & 0.169\\
groupGrade5:grapheme\_complexity & -0.025 & 0.065 & -0.151 & 0.100 & 1.000 & 0.137\\
frequency:grapheme\_complexity & -0.051 & 0.089 & -0.232 & 0.129 & 1.000 & 0.208\\
groupGrade3:graphomotor\_difficulty & -0.025 & 0.066 & -0.154 & 0.103 & 1.000 & 0.137\\
groupGrade5:graphomotor\_difficulty & -0.031 & 0.064 & -0.156 & 0.095 & 1.000 & 0.14\\
frequency:graphomotor\_difficulty & 0.158 & 0.091 & -0.024 & 0.337 & 1.000 & 0.78\\
grapheme\_complexity:graphomotor\_difficulty & 0.027 & 0.091 & -0.154 & 0.206 & 1.000 & 0.189\\
groupGrade3:frequency:grapheme\_complexity & 0.015 & 0.125 & -0.229 & 0.261 & 1.000 & 0.248\\
groupGrade5:frequency:grapheme\_complexity & 0.097 & 0.123 & -0.144 & 0.341 & 1.000 & 0.342\\
groupGrade3:frequency:graphomotor\_difficulty & -0.049 & 0.127 & -0.295 & 0.198 & 1.000 & 0.268\\
groupGrade5:frequency:graphomotor\_difficulty & -0.174 & 0.124 & -0.412 & 0.069 & 1.000 & 0.635\\
groupGrade3:grapheme\_complexity:graphomotor\_difficulty & -0.036 & 0.125 & -0.284 & 0.213 & 1.000 & 0.255\\
groupGrade5:grapheme\_complexity:graphomotor\_difficulty & -0.020 & 0.124 & -0.261 & 0.224 & 1.000 & 0.25\\
frequency:grapheme\_complexity:graphomotor\_difficulty & 0.070 & 0.166 & -0.249 & 0.397 & 1.000 & 0.357\\
groupGrade3:frequency:grapheme\_complexity:graphomotor\_difficulty & 0.044 & 0.227 & -0.395 & 0.487 & 1.000 & 0.457\\
groupGrade5:frequency:grapheme\_complexity:graphomotor\_difficulty & -0.086 & 0.222 & -0.523 & 0.351 & 1.000 & 0.465\\
\bottomrule
\addlinespace
\insertTableNotes
\end{longtable}

}

\end{lltable}

These estimations are better understood visually. Thus, we plot the predictions of this model against raw data in Figure \ref{fig:velocity-predictions-plot}.

\begin{Shaded}
\begin{Highlighting}[]
\CommentTok{\# retrieving the model\textquotesingle{}s predictions}
\NormalTok{velocity\_predictions }\OtherTok{\textless{}{-}}\NormalTok{ df2 }\SpecialCharTok{\%\textgreater{}\%}
    \FunctionTok{data\_grid}\NormalTok{(graphomotor\_difficulty, grapheme\_complexity, frequency, group) }\SpecialCharTok{\%\textgreater{}\%}
    \FunctionTok{cbind}\NormalTok{(., }\FunctionTok{fitted}\NormalTok{(}
        \AttributeTok{object =}\NormalTok{ mod2, }\AttributeTok{newdata =}\NormalTok{ ., }\AttributeTok{resp =} \StringTok{"meanvelocity"}\NormalTok{,}
        \AttributeTok{scale =} \StringTok{"response"}\NormalTok{, }\AttributeTok{probs =} \FunctionTok{c}\NormalTok{(}\FloatTok{0.025}\NormalTok{, }\FloatTok{0.975}\NormalTok{),}
        \AttributeTok{re\_formula =} \ConstantTok{NA}\NormalTok{, }\AttributeTok{robust =} \ConstantTok{TRUE}
\NormalTok{        ) ) }\SpecialCharTok{\%\textgreater{}\%}
\NormalTok{    ungroup }\SpecialCharTok{\%\textgreater{}\%}
\NormalTok{    dplyr}\SpecialCharTok{::}\FunctionTok{rename}\NormalTok{(}\AttributeTok{estimate =}\NormalTok{ Estimate, }\AttributeTok{mad =}\NormalTok{ Est.Error, }\AttributeTok{lower =}\NormalTok{ Q2}\FloatTok{.5}\NormalTok{, }\AttributeTok{upper =}\NormalTok{ Q97}\FloatTok{.5}\NormalTok{)}
\end{Highlighting}
\end{Shaded}

\begin{figure}[!htb]

{\centering \includegraphics[width=1\linewidth]{supplementary_materials_files/figure-latex/velocity-predictions-plot-1} 

}

\caption{Mean velocity by grade and word frequency (x-axis), grapheme complexity (in colour), and graphomotor difficulty (in panels). Transparent points represent individual data per participant. The surimposed dots and intervals represent the model's predictions (median and 95\% credible interval of the posterior distribution).}\label{fig:velocity-predictions-plot}
\end{figure}

As can be seen in Figure \ref{fig:velocity-predictions-plot}, the model most notably predicts higher velocity for hard letters as compared to easy letters, excepted for low frequency words in Grade 1. First graders seem to have lower velocity than third graders, who themselves seem to have lower velocity than fifth graders on average. As can be seen from Table \ref{tab:velocity-summary}, BFs favouring the alternative hypothesis (relative to the null hypothesis) are BFs for the difference between Grade 1 and Grade 3 (\(\beta\) = 0.562, 95\% CrI {[}0.496, 0.628{]}, BF\textsubscript{10} = 1.213 x 10\^{}15), as well as between Grade 1 and Grade 5 (\(\beta\) = 0.793, 95\% CrI {[}0.729, 0.859{]}, BF\textsubscript{10} = 1.859 x 10\^{}16), and the effect of graphomotor difficulty in Grade 1 (\(\beta\) = 0.159, 95\% CrI {[}0.066, 0.252{]}, BF\textsubscript{10} = 16.037). Predictions from this model for each condition are also summarised in Table \ref{tab:velocity-predictions-summary}.

\begin{table}[htb]

\begin{center}
\begin{threeparttable}

\caption{\label{tab:velocity-predictions-summary}Estimated mean velocity in each condition.}

\scriptsize{

\begin{tabular}{cccccccc}
\toprule
Group & \multicolumn{1}{c}{Frequency} & \multicolumn{1}{c}{Grapheme complexity} & \multicolumn{1}{c}{Graphomotor difficulty} & \multicolumn{1}{c}{Estimate} & \multicolumn{1}{c}{MAD} & \multicolumn{1}{c}{Lower} & \multicolumn{1}{c}{Upper}\\
\midrule
Grade1 & LF & Simple & t & 10.439 & 0.649 & 9.228 & 11.835\\
Grade1 & LF & Simple & f & 11.350 & 0.730 & 9.992 & 12.881\\
Grade1 & LF & Complex & t & 11.140 & 0.708 & 9.836 & 12.636\\
Grade1 & LF & Complex & f & 12.020 & 0.793 & 10.538 & 13.677\\
Grade1 & HF & Simple & t & 10.576 & 0.676 & 9.321 & 12.025\\
Grade1 & HF & Simple & f & 12.975 & 0.831 & 11.427 & 14.742\\
Grade1 & HF & Complex & t & 10.362 & 0.661 & 9.131 & 11.763\\
Grade1 & HF & Complex & f & 13.521 & 0.855 & 11.928 & 15.358\\
Grade3 & LF & Simple & t & 19.061 & 1.367 & 16.534 & 21.957\\
Grade3 & LF & Simple & f & 21.336 & 1.525 & 18.496 & 24.615\\
Grade3 & LF & Complex & t & 19.793 & 1.427 & 17.149 & 22.855\\
Grade3 & LF & Complex & f & 20.730 & 1.487 & 18.022 & 23.867\\
Grade3 & HF & Simple & t & 18.659 & 1.332 & 16.159 & 21.537\\
Grade3 & HF & Simple & f & 21.965 & 1.535 & 19.107 & 25.269\\
Grade3 & HF & Complex & t & 17.680 & 1.288 & 15.295 & 20.402\\
Grade3 & HF & Complex & f & 21.822 & 1.544 & 18.957 & 25.160\\
Grade5 & LF & Simple & t & 23.643 & 1.627 & 20.601 & 27.133\\
Grade5 & LF & Simple & f & 26.864 & 1.876 & 23.411 & 30.888\\
Grade5 & LF & Complex & t & 23.158 & 1.610 & 20.167 & 26.606\\
Grade5 & LF & Complex & f & 26.743 & 1.868 & 23.308 & 30.681\\
Grade5 & HF & Simple & t & 23.599 & 1.649 & 20.547 & 27.100\\
Grade5 & HF & Simple & f & 26.604 & 1.844 & 23.236 & 30.499\\
Grade5 & HF & Complex & t & 24.416 & 1.695 & 21.327 & 27.982\\
Grade5 & HF & Complex & f & 27.526 & 1.937 & 23.944 & 31.594\\
\bottomrule
\addlinespace
\end{tabular}

}

\begin{tablenotes}[para]
\normalsize{\textit{Note.} For each condition, the 'Estimate' and 'MAD' columns contain the
    median and the median absolute deviation (MAD) of the posterior distribution,
    respectively. The 'Lower' and 'Upper' columns contain the lower and upper
    bounds of the 95\% credible interval.}
\end{tablenotes}

\end{threeparttable}
\end{center}

\end{table}

\hypertarget{letter-size}{%
\subsection{Letter size}\label{letter-size}}

Table \ref{tab:size-summary} reports the estimates (median of the posterior distribution) and associated 95\% credible intervals and \(\text{BF}\)s for all parameters regarding the letter size.

\begin{lltable}

\begin{TableNotes}[para]
\normalsize{\textit{Note.} For each slope (for each line), the first two columns represent the
    estimated most probable value and its standard error (SE). The 'Lower' and
    'Upper' columns contain the lower and upper bounds of the 95\% CrI, whereas
    the 'Rhat' column reports the Gelman-Rubin statistic. The last column reports
    the Bayes factor in favour of the alternative hypothesis, relative to the
    null hypothesis (BF10).}
\end{TableNotes}

\scriptsize{

\begin{longtable}{ccccccc}\noalign{\getlongtablewidth\global\LTcapwidth=\longtablewidth}
\caption{\label{tab:size-summary}Estimates and BFs for the slopes for the letter size.}\\
\toprule
Term & \multicolumn{1}{c}{Estimate} & \multicolumn{1}{c}{MAD} & \multicolumn{1}{c}{Lower} & \multicolumn{1}{c}{Upper} & \multicolumn{1}{c}{Rhat} & \multicolumn{1}{c}{BF10}\\
\midrule
\endfirsthead
\caption*{\normalfont{Table \ref{tab:size-summary} continued}}\\
\toprule
Term & \multicolumn{1}{c}{Estimate} & \multicolumn{1}{c}{MAD} & \multicolumn{1}{c}{Lower} & \multicolumn{1}{c}{Upper} & \multicolumn{1}{c}{Rhat} & \multicolumn{1}{c}{BF10}\\
\midrule
\endhead
Intercept & 2.289 & 0.023 & 2.241 & 2.333 & 1.000 & NA\\
groupGrade3 & -0.125 & 0.029 & -0.181 & -0.070 & 1.000 & 7.090 x 10\textasciicircum{}3\\
groupGrade5 & 0.006 & 0.028 & -0.049 & 0.061 & 1.000 & 0.059\\
frequency & 0.023 & 0.042 & -0.059 & 0.107 & 1.000 & 0.097\\
grapheme\_complexity & 0.032 & 0.042 & -0.051 & 0.116 & 1.000 & 0.115\\
graphomotor\_difficulty & 0.703 & 0.042 & 0.618 & 0.788 & 1.000 & -9.629 x 10\textasciicircum{}16\\
groupGrade3:frequency & -0.021 & 0.057 & -0.133 & 0.090 & 1.000 & 0.124\\
groupGrade5:frequency & -0.004 & 0.055 & -0.112 & 0.105 & 1.000 & 0.112\\
groupGrade3:grapheme\_complexity & -0.027 & 0.056 & -0.138 & 0.085 & 1.000 & 0.126\\
groupGrade5:grapheme\_complexity & -0.012 & 0.055 & -0.121 & 0.097 & 1.000 & 0.111\\
frequency:grapheme\_complexity & 0.023 & 0.081 & -0.139 & 0.186 & 1.000 & 0.17\\
groupGrade3:graphomotor\_difficulty & -0.115 & 0.057 & -0.225 & -0.003 & 1.000 & 0.843\\
groupGrade5:graphomotor\_difficulty & -0.193 & 0.055 & -0.303 & -0.084 & 1.000 & 42.547\\
frequency:graphomotor\_difficulty & 0.074 & 0.081 & -0.090 & 0.235 & 1.000 & 0.251\\
grapheme\_complexity:graphomotor\_difficulty & 0.022 & 0.081 & -0.142 & 0.186 & 1.000 & 0.169\\
groupGrade3:frequency:grapheme\_complexity & 0.004 & 0.109 & -0.208 & 0.220 & 1.000 & 0.222\\
groupGrade5:frequency:grapheme\_complexity & -0.048 & 0.108 & -0.260 & 0.164 & 1.000 & 0.237\\
groupGrade3:frequency:graphomotor\_difficulty & -0.041 & 0.110 & -0.257 & 0.177 & 1.000 & 0.237\\
groupGrade5:frequency:graphomotor\_difficulty & -0.102 & 0.108 & -0.316 & 0.108 & 1.000 & 0.334\\
groupGrade3:grapheme\_complexity:graphomotor\_difficulty & 0.012 & 0.111 & -0.207 & 0.228 & 1.000 & 0.228\\
groupGrade5:grapheme\_complexity:graphomotor\_difficulty & 0.030 & 0.109 & -0.185 & 0.244 & 1.000 & 0.229\\
frequency:grapheme\_complexity:graphomotor\_difficulty & 0.053 & 0.150 & -0.248 & 0.352 & 1.000 & 0.324\\
groupGrade3:frequency:grapheme\_complexity:graphomotor\_difficulty & -0.025 & 0.204 & -0.423 & 0.370 & 1.000 & 0.408\\
groupGrade5:frequency:grapheme\_complexity:graphomotor\_difficulty & 0.023 & 0.201 & -0.369 & 0.414 & 1.000 & 0.409\\
\bottomrule
\addlinespace
\insertTableNotes
\end{longtable}

}

\end{lltable}

These estimations are better understood visually. Thus, we plot the predictions of this model against raw data in Figure \ref{fig:size-predictions-plot}.

\begin{Shaded}
\begin{Highlighting}[]
\CommentTok{\# retrieving the model\textquotesingle{}s predictions}
\NormalTok{size\_predictions }\OtherTok{\textless{}{-}}\NormalTok{ df2 }\SpecialCharTok{\%\textgreater{}\%}
    \FunctionTok{data\_grid}\NormalTok{(graphomotor\_difficulty, grapheme\_complexity, frequency, group) }\SpecialCharTok{\%\textgreater{}\%}
    \FunctionTok{cbind}\NormalTok{(., }\FunctionTok{fitted}\NormalTok{(}
        \AttributeTok{object =}\NormalTok{ mod2, }\AttributeTok{newdata =}\NormalTok{ ., }\AttributeTok{resp =} \StringTok{"lettersize"}\NormalTok{,}
        \AttributeTok{scale =} \StringTok{"response"}\NormalTok{, }\AttributeTok{probs =} \FunctionTok{c}\NormalTok{(}\FloatTok{0.025}\NormalTok{, }\FloatTok{0.975}\NormalTok{),}
        \AttributeTok{re\_formula =} \ConstantTok{NA}\NormalTok{, }\AttributeTok{robust =} \ConstantTok{TRUE}
\NormalTok{        ) ) }\SpecialCharTok{\%\textgreater{}\%}
\NormalTok{    ungroup }\SpecialCharTok{\%\textgreater{}\%}
\NormalTok{    dplyr}\SpecialCharTok{::}\FunctionTok{rename}\NormalTok{(}\AttributeTok{estimate =}\NormalTok{ Estimate, }\AttributeTok{mad =}\NormalTok{ Est.Error, }\AttributeTok{lower =}\NormalTok{ Q2}\FloatTok{.5}\NormalTok{, }\AttributeTok{upper =}\NormalTok{ Q97}\FloatTok{.5}\NormalTok{)}
\end{Highlighting}
\end{Shaded}

\begin{figure}[!htb]

{\centering \includegraphics[width=1\linewidth]{supplementary_materials_files/figure-latex/size-predictions-plot-1} 

}

\caption{Letter size by grade and word frequency (x-axis), grapheme complexity (in colour), and graphomotor difficulty (in panels). Transparent points represent individual data per participant. The surimposed dots and intervals represent the model's predictions (median and 95\% credible interval of the posterior distribution).}\label{fig:size-predictions-plot}
\end{figure}

As can be seen in Figure \ref{fig:size-predictions-plot}, the production of difficult letters was associated with greater letter size than the production of easy letters for all grades. As can be seen from Table \ref{tab:size-summary}, BFs favouring the alternative hypothesis (relative to the null hypothesis) are BFs for the difference between Grade 1 and Grade 3 (\(\beta\) = -0.125, 95\% CrI {[}-0.181, -0.07{]}, BF\textsubscript{10} = 7.090 x 10\^{}3), the effect of graphomotor difficulty in Grade 1 (\(\beta\) = 0.703, 95\% CrI {[}0.618, 0.788{]}, BF\textsubscript{10} = -9.629 x 10\^{}16), and the effect of graphomotor difficulty in Grade 5 (\(\beta\) = -0.193, 95\% CrI {[}-0.303, -0.084{]}, BF\textsubscript{10} = 42.547). Predictions from this model for each condition are also summarised in Table \ref{tab:size-predictions-summary}.

\begin{table}[htb]

\begin{center}
\begin{threeparttable}

\caption{\label{tab:size-predictions-summary}Estimated letter size in each condition.}

\scriptsize{

\begin{tabular}{cccccccc}
\toprule
Group & \multicolumn{1}{c}{Frequency} & \multicolumn{1}{c}{Grapheme complexity} & \multicolumn{1}{c}{Graphomotor difficulty} & \multicolumn{1}{c}{Estimate} & \multicolumn{1}{c}{MAD} & \multicolumn{1}{c}{Lower} & \multicolumn{1}{c}{Upper}\\
\midrule
Grade1 & LF & Simple & t & 7.489 & 0.427 & 6.691 & 8.391\\
Grade1 & LF & Simple & f & 14.538 & 0.837 & 12.946 & 16.292\\
Grade1 & LF & Complex & t & 7.665 & 0.436 & 6.838 & 8.586\\
Grade1 & LF & Complex & f & 14.815 & 0.872 & 13.169 & 16.659\\
Grade1 & HF & Simple & t & 7.405 & 0.429 & 6.614 & 8.310\\
Grade1 & HF & Simple & f & 15.058 & 0.852 & 13.431 & 16.889\\
Grade1 & HF & Complex & t & 7.550 & 0.430 & 6.735 & 8.483\\
Grade1 & HF & Complex & f & 16.120 & 0.922 & 14.372 & 18.056\\
Grade3 & LF & Simple & t & 7.147 & 0.456 & 6.305 & 8.115\\
Grade3 & LF & Simple & f & 12.482 & 0.793 & 10.983 & 14.186\\
Grade3 & LF & Complex & t & 7.022 & 0.449 & 6.197 & 7.969\\
Grade3 & LF & Complex & f & 12.503 & 0.787 & 11.021 & 14.182\\
Grade3 & HF & Simple & t & 6.998 & 0.434 & 6.180 & 7.962\\
Grade3 & HF & Simple & f & 12.454 & 0.786 & 11.002 & 14.111\\
Grade3 & HF & Complex & t & 6.972 & 0.446 & 6.139 & 7.925\\
Grade3 & HF & Complex & f & 13.006 & 0.832 & 11.461 & 14.752\\
Grade5 & LF & Simple & t & 8.081 & 0.498 & 7.157 & 9.157\\
Grade5 & LF & Simple & f & 13.509 & 0.824 & 11.949 & 15.281\\
Grade5 & LF & Complex & t & 8.288 & 0.506 & 7.343 & 9.378\\
Grade5 & LF & Complex & f & 14.054 & 0.873 & 12.399 & 15.896\\
Grade5 & HF & Simple & t & 8.625 & 0.525 & 7.638 & 9.761\\
Grade5 & HF & Simple & f & 13.495 & 0.827 & 11.942 & 15.235\\
Grade5 & HF & Complex & t & 8.310 & 0.515 & 7.359 & 9.412\\
Grade5 & HF & Complex & f & 14.209 & 0.884 & 12.551 & 16.088\\
\bottomrule
\addlinespace
\end{tabular}

}

\begin{tablenotes}[para]
\normalsize{\textit{Note.} For each condition, the 'Estimate' and 'MAD' columns contain the
    median and the median absolute deviation (MAD) of the posterior distribution,
    respectively. The 'Lower' and 'Upper' columns contain the lower and upper
    bounds of the 95\% credible interval.}
\end{tablenotes}

\end{threeparttable}
\end{center}

\end{table}

\hypertarget{acknowledgments}{%
\section{Acknowledgments}\label{acknowledgments}}

This work, carried out within the Labex BLRI (ANR-11-LABX-0036) and the Institut Convergence ILCB (ANR-16-CONV-0002), has benefited from support from the French Government, managed by the French National Agency for Research (ANR), under the project title DYSTACMAP (ANR-13-APPR-0010).

\newpage

\hypertarget{session-information}{%
\section{Session information}\label{session-information}}

\begin{Shaded}
\begin{Highlighting}[]
\FunctionTok{sessionInfo}\NormalTok{()}
\end{Highlighting}
\end{Shaded}

\begin{verbatim}
## R version 4.1.1 (2021-08-10)
## Platform: x86_64-apple-darwin17.0 (64-bit)
## Running under: macOS Big Sur 10.16
## 
## Matrix products: default
## BLAS:   /Library/Frameworks/R.framework/Versions/4.1/Resources/lib/libRblas.0.dylib
## LAPACK: /Library/Frameworks/R.framework/Versions/4.1/Resources/lib/libRlapack.dylib
## 
## locale:
## [1] en_US.UTF-8/en_US.UTF-8/en_US.UTF-8/C/en_US.UTF-8/en_US.UTF-8
## 
## attached base packages:
## [1] stats     graphics  grDevices utils     datasets  methods   base     
## 
## other attached packages:
##  [1] brms_2.16.2       Rcpp_1.0.7        BEST_0.5.3        HDInterval_0.2.2 
##  [5] glue_1.4.2        knitr_1.36        papaja_0.1.0.9997 readxl_1.3.1     
##  [9] GGally_2.1.2      modelr_0.1.8      posterior_1.1.0   patchwork_1.1.1  
## [13] forcats_0.5.1     stringr_1.4.0     dplyr_1.0.7       purrr_0.3.4      
## [17] readr_2.0.2       tidyr_1.1.4       tibble_3.1.5      tidyverse_1.3.1  
## [21] ggbeeswarm_0.6.0  ggplot2_3.3.5     extraDistr_1.9.1 
## 
## loaded via a namespace (and not attached):
##   [1] backports_1.2.1      plyr_1.8.6           igraph_1.2.6        
##   [4] splines_4.1.1        crosstalk_1.1.1      rstantools_2.1.1    
##   [7] inline_0.3.19        digest_0.6.28        htmltools_0.5.2     
##  [10] rsconnect_0.8.24     fansi_0.5.0          magrittr_2.0.1      
##  [13] checkmate_2.0.0      tzdb_0.1.2           RcppParallel_5.1.4  
##  [16] matrixStats_0.61.0   xts_0.12.1           prettyunits_1.1.1   
##  [19] colorspace_2.0-2     rvest_1.0.1          haven_2.4.3         
##  [22] xfun_0.26            callr_3.7.0          crayon_1.4.1        
##  [25] jsonlite_1.7.2       lme4_1.1-27.1        zoo_1.8-9           
##  [28] gtable_0.3.0         emmeans_1.7.0        V8_3.4.2            
##  [31] distributional_0.2.2 pkgbuild_1.2.0       rstan_2.26.3        
##  [34] abind_1.4-5          scales_1.1.1         mvtnorm_1.1-3       
##  [37] DBI_1.1.1            miniUI_0.1.1.1       xtable_1.8-4        
##  [40] stats4_4.1.1         StanHeaders_2.26.3   DT_0.19             
##  [43] htmlwidgets_1.5.4    httr_1.4.2           threejs_0.3.3       
##  [46] RColorBrewer_1.1-2   ellipsis_0.3.2       pkgconfig_2.0.3     
##  [49] reshape_0.8.8        loo_2.4.1            farver_2.1.0        
##  [52] dbplyr_2.1.1         utf8_1.2.2           tidyselect_1.1.1    
##  [55] rlang_0.4.11         reshape2_1.4.4       later_1.3.0         
##  [58] munsell_0.5.0        cellranger_1.1.0     tools_4.1.1         
##  [61] cli_3.0.1            generics_0.1.0       broom_0.7.9         
##  [64] ggridges_0.5.3       evaluate_0.14        fastmap_1.1.0       
##  [67] yaml_2.2.1           processx_3.5.2       fs_1.5.0            
##  [70] nlme_3.1-152         mime_0.12            projpred_2.0.2      
##  [73] xml2_1.3.2           compiler_4.1.1       bayesplot_1.8.1     
##  [76] shinythemes_1.2.0    rstudioapi_0.13      gamm4_0.2-6         
##  [79] beeswarm_0.4.0       curl_4.3.2           reprex_2.0.1        
##  [82] stringi_1.7.5        ps_1.6.0             Brobdingnag_1.2-6   
##  [85] lattice_0.20-44      Matrix_1.3-4         nloptr_1.2.2.2      
##  [88] markdown_1.1         shinyjs_2.0.0        tensorA_0.36.2      
##  [91] vctrs_0.3.8          pillar_1.6.3         lifecycle_1.0.1     
##  [94] bridgesampling_1.1-2 estimability_1.3     httpuv_1.6.3        
##  [97] R6_2.5.1             bookdown_0.24        promises_1.2.0.1    
## [100] gridExtra_2.3        vipor_0.4.5          rjags_4-11          
## [103] codetools_0.2-18     boot_1.3-28          MASS_7.3-54         
## [106] colourpicker_1.1.1   gtools_3.9.2         assertthat_0.2.1    
## [109] withr_2.4.2          shinystan_2.5.0      mgcv_1.8-36         
## [112] parallel_4.1.1       hms_1.1.1            grid_4.1.1          
## [115] minqa_1.2.4          coda_0.19-4          rmarkdown_2.11      
## [118] shiny_1.7.1          lubridate_1.8.0      base64enc_0.1-3     
## [121] dygraphs_1.1.1.6
\end{verbatim}

\newpage

\hypertarget{references}{%
\section*{References}\label{references}}
\addcontentsline{toc}{section}{References}

\hypertarget{refs}{}
\begin{CSLReferences}{1}{0}
\leavevmode\vadjust pre{\hypertarget{ref-R-papaja}{}}%
Aust, F., \& Barth, M. (2020). \emph{{papaja}: {Create} {APA} manuscripts with {R Markdown}}. Retrieved from \url{https://github.com/crsh/papaja}

\leavevmode\vadjust pre{\hypertarget{ref-R-brms_a}{}}%
Bürkner, P.-C. (2017). {brms}: An {R} package for {Bayesian} multilevel models using {Stan}. \emph{Journal of Statistical Software}, \emph{80}(1), 1--28. \url{https://doi.org/10.18637/jss.v080.i01}

\leavevmode\vadjust pre{\hypertarget{ref-forstmann_sequential_2016}{}}%
Forstmann, B. U., Ratcliff, R., \& Wagenmakers, E.-J. (2016). Sequential sampling models in cognitive neuroscience: Advantages, applications, and extensions. \emph{Annual Review of Psychology}, \emph{67}, 641--666. \url{https://doi.org/10.1146/annurev-psych-122414-033645}

\leavevmode\vadjust pre{\hypertarget{ref-gabry_visualization_2019}{}}%
Gabry, J., Simpson, D., Vehtari, A., Betancourt, M., \& Gelman, A. (2019). Visualization in bayesian workflow. \emph{Journal of the Royal Statistical Society: Series A (Statistics in Society)}, \emph{182}(2), 389--402. \url{https://doi.org/10.1111/rssa.12378}

\leavevmode\vadjust pre{\hypertarget{ref-gelman_bayesian_2020}{}}%
Gelman, A., Vehtari, A., Simpson, D., Margossian, C. C., Carpenter, B., Yao, Y., \ldots{} Modrák, M. (2020). Bayesian workflow. \emph{{arXiv}:2011.01808 {[}Stat{]}}. Retrieved from \url{http://arxiv.org/abs/2011.01808}

\leavevmode\vadjust pre{\hypertarget{ref-nalborczyk_introduction_2019}{}}%
Nalborczyk, L., Batailler, C., Lœvenbruck, H., Vilain, A., \& Bürkner, P.-C. (2019). An introduction to bayesian multilevel models using brms: A case study of gender effects on vowel variability in standard indonesian. \emph{Journal of Speech Language and Hearing Research}, \emph{62}(5), 1225--1242. \url{https://doi.org/10.1044/2018_JSLHR-S-18-0006}

\leavevmode\vadjust pre{\hypertarget{ref-R-base}{}}%
R Core Team. (2021). \emph{R: A language and environment for statistical computing}. Vienna, Austria: R Foundation for Statistical Computing. Retrieved from \url{https://www.R-project.org/}

\leavevmode\vadjust pre{\hypertarget{ref-wagenmakers_bayesian_2010}{}}%
Wagenmakers, E.-J., Lodewyckx, T., Kuriyal, H., \& Grasman, R. (2010). Bayesian hypothesis testing for psychologists: A tutorial on the savage--dickey method. \emph{Cognitive Psychology}, \emph{60}(3), 158--189. \url{https://doi.org/10.1016/j.cogpsych.2009.12.001}

\leavevmode\vadjust pre{\hypertarget{ref-R-tidyverse}{}}%
Wickham, H., Averick, M., Bryan, J., Chang, W., McGowan, L. D., François, R., \ldots{} Yutani, H. (2019). Welcome to the {tidyverse}. \emph{Journal of Open Source Software}, \emph{4}(43), 1686. \url{https://doi.org/10.21105/joss.01686}

\leavevmode\vadjust pre{\hypertarget{ref-winter_poisson_2021}{}}%
Winter, B., \& Bürkner, P.-. C. (2021). \emph{Poisson regression for linguists: A tutorial introduction to modeling count data with brms}. preprint, preprint. \url{https://doi.org/10.31219/osf.io/93kaf}

\leavevmode\vadjust pre{\hypertarget{ref-R-knitr}{}}%
Xie, Y. (2015). \emph{Dynamic documents with {R} and knitr} (2nd ed.). Boca Raton, Florida: Chapman; Hall/CRC. Retrieved from \url{https://yihui.org/knitr/}

\end{CSLReferences}


\end{document}
